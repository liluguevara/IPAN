\documentclass[11pt]{article}
\usepackage{amsmath,amsfonts,amsthm,amssymb}
\usepackage[spanish]{babel}
\usepackage{graphicx}
\usepackage[a4paper]{geometry}
\usepackage[latin1]{inputenc}
\usepackage{subfigure}
%\usepackage{enumitem}
\usepackage{textcomp}
\usepackage{listings}
\lstset{language=Matlab, breaklines=true, basicstyle=\footnotesize}
%\lstset{numbers=left, numberstyle=\tiny, stepnumber=1, numbersep=2pt}

\geometry{left=2cm,right=2cm,top=2cm,bottom=2cm}

\begin{document}
\renewcommand{\labelitemi}{$\bullet$}
\begin{center}
\large{{\bf Introducci\'on a la Programaci\'on y An\'alisis Num\'erico\\[5pt] A\~no 2022}}\\[5pt]
\large{{\bf Pr\'actica 9: Sistemas de Ecuaciones Lineales.}}\\
\rule{17cm}{1pt}
\bigskip
\end{center}
{\bf Ej. 1: M\'etodos directos: Eliminaci\'on Gaussiana{/} M�todo de Gauss-Jordan. Pivoteo Parcial}
\begin{itemize}
\item[{\bf a)}] Considere los siguientes sistemas de ecuaciones $Ax=b$, donde
\[
A_1=
  \begin{bmatrix}
    0 & 1  & -1 \\
    2 & -1 & -1 \\
    1 & 1  & -1
  \end{bmatrix},
b_1=
\begin{bmatrix}
 2 \\
 0 \\
 1
\end{bmatrix}
\]
\[
A_2=
  \begin{bmatrix}
    1 &  0 & 1 \\
    1 &  1 & 1 \\
    1 & -1 & 1
  \end{bmatrix},
b_2=
\begin{bmatrix}
 2 \\
 3 \\
 1
\end{bmatrix}.
\]
\begin{itemize}

 \item[i.] Para ambos sistemas, calcule el determinante de $A$ haciendo uso de la funci�n {\ttfamily det()} de MATLAB/OCTAVE. ?`Qu\'e puede decir del sistema $A_2$?

 \item[ii.] ?`Para cu\'al de los sistema puede asegurar la existencia de una soluci\'on \'unica? Resuelva el sistema de ecuaciones mediante eliminaci�n Gaussiana con \textit{pivoteo parcial}.

%\item[iii.] Realice un programa en MATLAB/OCTAVE que resuelva un sistema de ecuaciones mediante el m\'etodo de eliminaci\'on \textit{Gaussiana}.

%\item[iV.] ?`Cu\'antas operaciones de punto flotante realiza este algoritmo para hallar la soluci\'on a $Ax=b$ si la matriz del sistema de $N\times N$ es no-singular? Compare con las necesarias para hallar la soluci\'on mediante el c\'alculo expl\'icito de $A^{-1}$. Recordar que $A^{-1}=\frac{[Adj(A)]^T}{det(A)}$ donde $Adj(A)$ es la matriz adjunta.
\end{itemize}

\item[{\bf b)}] {\bf Estimaci\'on a-priori del error. An\'alisis perturbativo}:

Dado el sistema $Ax = b$, suponga que el t\'ermino $b$ se perturba por una cantidad $\delta b$.


\noindent El efecto de la perturbaci\'on sobre la soluci\'on ser\'a $x + \delta x$, con $\delta x$ acotada por
\begin{equation*}
 \frac{\parallel \delta x\parallel}{\parallel x\parallel }\leq cond(A)\frac{\parallel \delta b \parallel}{\parallel b\parallel}
\end{equation*}

\noindent Si la matriz $A$ es la que es perturbada por una cantidad $\delta A$, el efecto de tal perturbaci\'on sobre la soluci\'on ser\'a $x + \delta x$ con $\delta x$ acotada por
\begin{equation*}
 \frac{\parallel \delta x\parallel}{\parallel x+\delta x\parallel }\leq cond(A)\frac{\parallel \delta A \parallel}{\parallel A\parallel}
\end{equation*}

\noindent Inspeccione la funci\'on {\ttfamily cond()} de MATLAB/OCTAVE. Calcule los valores de {\ttfamily cond(A1)} y {\ttfamily cond(A2)}. ?`Qu\'e indican estos resultados?



%\item[{\bf c)}] El sistema
%\[
%\begin{bmatrix}
%0.0003 & 1.566 \\
%0.3454 & -2.436
%\end{bmatrix}
%\begin{bmatrix}
%x_1 \\ x_2
%\end{bmatrix}
%=
%\begin{bmatrix}
%1.569 \\ 1.018
%\end{bmatrix},
%\]
%tiene soluci\'on exacta dada por $x=[10$,$1]^t$. Utilice la aritm\'etica de decimal de cuatro d\'igitos para resolverlo sin y con estrategia de pivoteo parcial. Indique el porque de la diferencia en las soluciones num\'ericas obtenidas.

\item[{\bf c)}] Sean los sistemas de ecuaciones $Ax=b$ determinados por:
\[
A_1=
  \begin{bmatrix}
    10^{-15} & 1      \\
    1        & 10^{11}
  \end{bmatrix},
A_2=
\begin{bmatrix}
    10^{-14.6} & 1      \\
    1        & 10^{15}
\end{bmatrix},
A_3=
\begin{bmatrix}
    10^{15} & 1      \\
    1      & 10^{-14.6}
\end{bmatrix}
\]
y
\[
b_1=
\begin{bmatrix}
 1+10^{-15} \\
 10^{11}+1
\end{bmatrix},
b_2=
\begin{bmatrix}
 1+10^{-14.6} \\
 10^{15}+1
\end{bmatrix},
b_3=
\begin{bmatrix}
 10^{15}+1 \\
 1+10^{-14.6}
\end{bmatrix}.
\]

\noindent Calcule los determinantes de las matrices $A$ para cada sistema. Conociendo los determinantes de dichas matrices y sin resolver los sistemas, ?`qu\'e puede decir de las soluciones de los mismos? ?`Qu\'e puede decir a partir del n\'umero de condici\'on?


%\item[{\bf e)}] Se sabe que que la soluci\'on de los sistemas del inciso anterior es $x=[1,$ $1]$. Utilice las estrategias de \textit{pivoteo parcial escalado} y \textit{pivoteo total}, seg\'un sea conveniente, para aproximar la soluci�n de dichos sistemas. ?`Qu\'e observa? ?`Qu\'e conclusiones puede sacar de dichos resultados acerca de estas estrategias?

%\item[{\bf f)}] La matriz de {\bf Hilbert} $\mathcal{H}$ representa la soluci�n al problema de {\bf aproximaci�n por m�nimos cuadrados}, que estudiaremos en la {\bf Unidad 5}. La misma es de la forma
%\begin{equation*}
%\mathcal{H}=[h_{mn}]=\frac{1}{m+n-1}
%\end{equation*}
%Genere una funci\'on en MATLAB/OCTAVE que genere una matriz $\mathcal{H}$ arbitraria de dimensi\'on $M\times N$ y calcule el n\'umero de condici\'on para aquellas que resultan cuadradas. Grafique {\ttfamily cond(H)} en funci\'on de $N=1\dots 12$.

%\item[{\bf g)}] Considere el sistema de ecuaciones dado por
%\[
%\begin{bmatrix}
%-1 & -2 & 2  \\
%1  &  1 & -1 \\
%1  &  2 & -1 \\
%\end{bmatrix}
%\begin{bmatrix}
%x_1 \\ x_2 \\ x_3
%\end{bmatrix}
%=
%\begin{bmatrix}
%-1 \\ 1 \\ 2
%\end{bmatrix}.
%\]

%\begin{itemize}
% \item[i.] Construya la matriz aumentada para dicho sistema y resuelva utilizando el m\'etodo de \textit{Gauss-Jordan}.

% \item[ii.] Modifique el programa desarrollado en el inciso a) para encontrar las soluciones a un sistema de ecuaciones mediante el m\'etodo de \textit{Gauss-Jordan}.
%\end{itemize}
\end{itemize}
\medskip
\vspace{2cm}
{\bf Ej. 2: Factorizaci\'on \textit{LU}}

\begin{itemize}

\item[{\bf a)}] Encuentre algebraicamente la factorizaci\'on $PA = LU$ de las siguientes matrices:
\[
A=
  \begin{bmatrix}
    1  & 2 & -1 \\
    3  & 6 &  2 \\
    -1 & 1 & 4
  \end{bmatrix},
B=
  \begin{bmatrix}
    0 & 1 & 4 \\
    -1 & 2 & 1 \\
    1 & 3 & 3
  \end{bmatrix}.
\]
\item[{\bf b)}] De las l�neas de c�digo proporcionadas al final de la pr�ctica, utilice las necesarias para implementar como funci�n la resoluci�n de un sistema de ecuaciones lineal $Ax=b$ mediante factorizaci\'on \textit{LU} y sustituci�n hacia adelante y hacia atr�s. El programa debe devolver como variables de salida las matrices $L$, $U$, $P$ y el vector soluci\'on $x$.	
\item[{\bf c)}] Utilizando el programa implementado en el inciso anterior, encuentre la factorizaci\'on \textit{LU} y la soluci\'on de los sistemas de ecuaciones $Ax=b$ para las matrices de coeficientes:
\[
A_1=
  \begin{bmatrix}
    0 & 1 & 2 \\
    1 & 2 & 3 \\
    2 & 3 & 2
  \end{bmatrix},
%\]
%\[
A_2=
  \begin{bmatrix}
    2 & 1 & 0 \\
    0 & 3 & 2 \\
    1 & 2 & 4
  \end{bmatrix},
%\]
%\[
A_3=
  \begin{bmatrix}
    1 & 2 & 1 \\
    2 & 0 & -2 \\
    -1 & 2 & 3
  \end{bmatrix}
\]
y
\[
b_1=
  \begin{bmatrix}
    2 \\
    4 \\
    5
  \end{bmatrix},
b_2=
  \begin{bmatrix}
    4 \\
    12 \\
    17
  \end{bmatrix},
b_3=
  \begin{bmatrix}
   1 \\
   -2 \\
   3
  \end{bmatrix}
\]
%\item[{\bf d)}] Sea la siguiente matriz sim\'etrica, definida positiva,
%\[
%A=
%  \begin{bmatrix}
%    38 & 3  & 4  & 6  & 5 \\
%    3  & 48 & 6  & 7  & 7 \\
%    4  & 6  & 52 & 1  & 6 \\
%    6  & 7  & 1  & 56 & 10 \\
%    5  & 7  & 6  & 10 & 74
%  \end{bmatrix}
%\]
%\begin{itemize}
% \item[i.] Desarrolle un programa en MATLAB/OCTAVE que ejecute una factorizaci\'on de \textit{Cholesky}.
% \item[ii.] Resolver los sistemas de ecuaciones $Ax = b$ donde
%\[
%b_1=
%  \begin{bmatrix}
%    56 \\
%    71 \\
%    69 \\
%    80 \\
%    102
%  \end{bmatrix},
%b_2=
%  \begin{bmatrix}
%    112 \\
%    142 \\
%    138 \\
%    160 \\
%    204
%  \end{bmatrix},
%b_3=
%  \begin{bmatrix}
%    168 \\
%    213 \\
%    207 \\
%    240 \\
%    306
%  \end{bmatrix}
%\]
%\end{itemize}
\item[{\bf d)}] MATLAB/OCTAVE cuenta con una funci�n predefinida que permite obtener la descomposici�n LU. La expresi�n \textbf{lu(A)} devuelve las matrices L y U o L, U y P, dependiendo de la cantidad de argumentos de salida explicitados. Utilice la funci�n predefinida para verificar las descomposiciones realizadas y mediante \textbf{y = L$\backslash$(P*b)} y \textbf{x = U$\backslash$y} determine el vector soluci\'on $x$. Compare con los resultados del inciso c). Las diferencias obtenidas, ?`a qu\'e considera que pueden deberse?
\item[{\bf e)}] Sea la siguiente matriz tri-diagonal $A$:
\[
A=
  \begin{bmatrix}
    2  & 2 & 0 & 0 & 0 & 0 \\\
    1  & 5 & 4 & 0 & 0 & 0  \\
    0  & 4 & 8 & 4 & 0 & 0  \\
    0  & 0 & 6 & 8 & 6 & 0  \\
    0  & 0 & 0 & 6 & 4 & 3  \\
    0  & 0 & 0 & 0 & 4 & 9
  \end{bmatrix}
\]
\begin{itemize}
 \item[i.] Realice una funci�n para su resoluci�n que explote la naturaleza tri-diagonal de la matriz, reduciendo as\'i el n\'umero de operaciones. \textbf{Ayuda}: revise la p�gina 306 del libro M�todos num�ricos para ingenieros, Chapra - Canale, Quinta edici�n.
 \item[ii.] Resolver los sistemas de ecuaciones $Ax = b$ donde
\[
b_1=
  \begin{bmatrix}
    4 \\
    10 \\
    16 \\
    20 \\
    13 \\
    13
\end{bmatrix},
b_2=
  \begin{bmatrix}
    8  \\
    20 \\
    32 \\
    40 \\
    26  \\
    26
  \end{bmatrix},
b_3=
  \begin{bmatrix}
    12 \\
    30 \\
    48 \\
    60 \\
    39 \\
    39
  \end{bmatrix}
\]
\end{itemize}
\end{itemize}
\medskip
{\bf Ej. 3: M\'etodos iterativos: Gauss-Seidel y Jacobi.}

\noindent M\'etodo de Jacobi: dada $A$ de $n\times n$ con elementos diagonales no nulos. A partir de un vector $\textbf{x}^{(0)}$ de orden $n$, se obtiene el vector soluci\'on de forma iterativa. Es decir, para $k= 0, 1, 2, ...$

\begin{equation} 
x_{i}^{(k)} = \frac{- \sum_{j=1 \atop j \neq i}^{n}a_{ij} x_{j}^{k-1} + b_{i}}{a_{ii}}, \quad \textsl{para i = 1, ... n} \nonumber 
\end{equation} 

\noindent M\'etodo de Gauss-Seidel: dada $A$ de $n\times n$ con elementos diagonales no nulos. A partir de un vector $\textbf{x}^{(0)}$ de orden $n$, se obtiene el vector soluci\'on de forma iterativa. Es decir, para $k= 0, 1, 2, ...$

\begin{equation} 
x_{i}^{(k)} = \frac{- \sum_{j=1}^{i-1}a_{ij} x_{j}^{k} -\sum_{j=i+1}^{n}a_{ij}x_{j}^{(k-1)} + b_{i}}{a_{ii}}, \quad \textsl{para i = 1, ... n} \nonumber 
\end{equation} 

En general como una primera aproximaci\'on suele tomarse el vector nulo, es decir, $\textbf{x}^{(0)}=0$. Adem\'as, como todo m\'etodo iterativo necesitan de un criterio de corte para las iteraciones. Utilizaremos $\frac{||\textbf{x}^{(k)}-\textbf{x}^{(k-1)}||}{||\textbf{x}^{(k)}||} < \epsilon$, para una cierta tolerancia $\epsilon$ preestablecida. 


\begin{itemize}

\item[{\bf a)}] Realice un programa en MATLAB/OCTAVE que permita resolver un sistema de ecuaciones del tipo $Ax=b$ mediante

 \begin{itemize}
  \item[i.] El m\'etodo iterativo de \textit{Jacobi}.
  \item[ii.] El m\'etodo iterativo de \textit{Gauss-Seidel}.
\end{itemize}
\item[{\bf b)}] Utilice los programas desarrollados en el inciso anterior para resolver el sistema:
\[
\begin{bmatrix}
3 & 2   \\
1 & 2
\end{bmatrix}
\begin{bmatrix}
x_1 \\ x_2
\end{bmatrix}
=
\begin{bmatrix}
1 \\ -1
\end{bmatrix}.
\]
cuya soluci\'on \'unica esta dada por $x=[1,-1]^T$
\item[{\bf c)}] Considere una barra de metal de $10$~m de longitud que es sometida a una fuente de calor en un extremo por un tiempo determinado. Al cabo de ese tiempo, la fuente de calor cesa y se miden temperaturas de $10^{\circ}$C y $200^{\circ}$C en ambos extremos. Se desea encontrar la temperatura en los puntos intermedios $x_1$, $x_2$, $x_3$ y $x_4$ ubicados a $2$~m equidistantes unos de otros. Asumiendo que la temperatura en cada punto es el promedio de las temperaturas medidas en los puntos vecinos,
\begin{itemize}
 \item[i.] Escriba el sistema de ecuaciones $Ax=b$ que representa el problema en cuesti\'on.
 \item[ii.] Encuentre las temperaturas en cada punto utilizando los m\'etodos de \textit{Jacobi} y \textit{Gauss-Seidel} programados en el inciso \textit{a}.
\end{itemize}
\item[{\bf d)}] Considere el sistema de ecuaciones dado, con soluci\'on exacta $x=(1,1)$:
\[
\begin{bmatrix}
1 & -5   \\
7 & -1
\end{bmatrix}
\begin{bmatrix}
x_1 \\ x_2
\end{bmatrix}
=
\begin{bmatrix}
-4 \\ 6
\end{bmatrix}.
\]

\begin{itemize}
 \item[i.] Utilizando la aproximaci\'on inicial $(x_1,x_2)=(0,0)$, resuelva el mismo sistema mediante los m\'etodos de \textit{Gauss-Seidel} y \textit{Jacobi}.
 \item[ii.] ?`Qu\'e puede decir de la convergencia de estos m\'etodos en este caso? ?`Por qu\'e ocurre este fen\'omeno.
\end{itemize}
\item[{\bf e)}] Sean los siguientes sistemas de ecuaciones:
\[
\begin{bmatrix}
3.8  & 1.6 & 0.9   \\
-0.7 & 5.4 & 1.6   \\
1.5 & 1.1 & 3.2
\end{bmatrix}
\begin{bmatrix}
x_1 \\ x_2 \\ x_3
\end{bmatrix}
=
\begin{bmatrix}
15.5 \\ 10.3 \\ 3.5
\end{bmatrix},
\begin{bmatrix}
1  & 0 & 1   \\
-1 & 1 & 0   \\
1  & 2 & -3
\end{bmatrix}
\begin{bmatrix}
x_1 \\ x_2 \\ x_3
\end{bmatrix}
=
\begin{bmatrix}
4 \\ 1 \\ -4
\end{bmatrix}.
\]
y
\[
 \begin{bmatrix}
1  & 0.5 & 0.5   \\
0.5 & 1 & 0.5   \\
0.5 & 0.5 & 1
\end{bmatrix}
\begin{bmatrix}
x_1 \\ x_2 \\ x_3
\end{bmatrix}
=
\begin{bmatrix}
2 \\ 2 \\ 2
\end{bmatrix}.
\]
\begin{itemize}
 \item[i.] Resuelva los sistemas de ecuaciones dados utilizando los m\'etodos de \textit{Gauss-Seidel} y \textit{Jacobi}. ?`Qu\'e puede decir de la convergencia de estos m\'etodos para los sistemas dados?
 %\item[ii.] Calcule el radio espectral $\rho(T)$ de las matrices $A_{1,2,3}$ para ambos m\'etodos y haga un an\'alisis de la convergencia (o divergencia) para los sistemas dados. \underline{\textit{Ayuda}}:
 %\begin{eqnarray*}
 %T_{J}&=&D^{-1}(L+U)\\
 %T_{GS}&=&(D-L)^{-1}U
 %\end{eqnarray*}Donde $A=D-L-U$, con $D$ la matriz diagonal formada con los elementos de la diagonal principal de $A$ y ceros en los dem\'as
%elementos, $-L$ la matriz estrictamente triangular inferior formada con los elementos de $A$ situados bajo la diagonal principal y $-U$ la matriz estrictamente triangular superior formada con los elementos situados por arriba de la diagonal principal.
\item[ii.]  Los elementos de la diagonal de la matriz nos proporcionan un criterio suficiente pero no necesario de convergencia. Si el elemento de la diagonal en m\'odulo es mayor a la suma de los m\'odulos de los dem\'as elementos de la fila, es decir, si se cumple:
\begin{equation}
|a_{i,i}| > \sum_{j=1 \atop j \neq i }^{n} |a_{i,j}| \nonumber
\end{equation}
la convergencia est\'a garantizada \footnote{Notar que al ser un criterio suficiente pero no necesario, los m\'etodos pueden converger a\'un cuando no se cumpla, pero la convergencia estar\'a garantizada cuando la condici\'on se satisfaga.}. Calcule dicho criterio para las matrices del inciso anterior.
\end{itemize}
\end{itemize}

%{\bf Ej. 3: Estimaci\'on a-posteriori del error.}
%\begin{itemize}
%\item[{\bf a)}] Sabiendo que la exactitud de la soluci\'on calculada depende del numero de condici\'on de la matriz seg\'un la expresi\'on:
%\begin{equation*}
% \frac{\parallel x-\hat{x}\parallel}{\parallel x\parallel }\leq cond(A)\frac{\parallel b-A\hat{x}\parallel}{\parallel b\parallel}
%\end{equation*}
%Dado el sistema de ecuaciones:
% \[
%\begin{bmatrix}
%0.89 & 0.53   \\
%0.47 & 0.28
%\end{bmatrix}
%\begin{bmatrix}
%x_1 \\ x_2
%\end{bmatrix}
%=
%\begin{bmatrix}
%0.36 \\ 0.19
%\end{bmatrix}.
%\]
%\begin{description}
% \item[i] Considerando que una soluci\'on aproximada $\hat{x}=(-11.5,20)$, calcule el vector residual $r=b-A\hat{x}$ y el error relativo si la soluci\'on exacta es $x=(1,-1)$.
%\end{description}
%\end{itemize}
\newpage

\noindent L�neas de c\'odigo para la factorizaci\'on $LU$ con pivoteo y para la resoluci�n del sistema de ecuaciones por sustituci\'on:
\begin{lstlisting}[language=Matlab]
% factorizacion PA = LU
% y resolucion del sistema mediante sustitucion hacia adelante y hacia atras
clc; clear; close all;

M = load('matriz.dat');
n = length(M(:,1));
A = M(:,1:3);
b = M(:,4);
P = eye(n);
L = eye(n);
U = zeros(n);

% permutacion de la matriz a:
% la permutacion se realiza mirando el maximo abs(value) por columna
for i = 1: n-1
  [val_a, index_a] = max(abs(A(i:n,i)));
  
  if(abs(val_a) <= 1E-8)
    disp('elemento nulo en la diagonal')
    stop
  else
    A([i index_a+i-1],:) = A([index_a+i-1,i], :);
    P([i index_a+i-1],:) = P([index_a+i-1,i],:);
   
  endif
endfor 

U(1,:) = A(1,:);
L(:,1) = A(:,1)./U(1,1);

for i = 2: n-1;
  U(i,i) = A(i,i) - sum((L(i,1:i-1))'.*U(1:i-1,i));
  for j = i+1: n;
    U(i,j) = A(i,j)-sum((L(i,1:i-1))'.*U(1:i-1,j));
    L(j,i) = 1/U(i,i)*(A(j,i)-sum((L(j,1:i-1))'.*U(1:i-1,i)));
  endfor
endfor
  
U(n,n) = A(n,n) - sum((L(n,1:n-1))'.*U(1:n-1,n));

% sustituci�n hacia adelante

b = P*b;
for i = 2: n
  suma = b(i);
    for j =1: i-1
      suma = suma - L(i,j)*b(j);
    end
    b(i) = suma;
end
  
% sustituci�n hacia atr�s

x(n)  = b(n)/U(n,n);
for i = n-1: -1: 1
  suma = 0;
  for j = i + 1: n
    suma = suma + U(i,j)*x(j);
  end
  x(i) = (b(i) - suma)/U(i,i);
end
\end{lstlisting}

\end{document}
