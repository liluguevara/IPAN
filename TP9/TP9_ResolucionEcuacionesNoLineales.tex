\documentclass[11pt]{article}
\usepackage{amsmath,amsfonts,amsthm,amssymb}
\usepackage[spanish]{babel}
\usepackage{graphicx}
\usepackage[a4paper]{geometry}
\usepackage[latin1]{inputenc}
\usepackage{subfigure}
%\usepackage{enumitem}
\usepackage{textcomp}
\usepackage{listings}
\usepackage{color}

\definecolor{mygreen}{rgb}{0,0.6,0}
\definecolor{mygray}{rgb}{0.5,0.5,0.5}
\definecolor{mymauve}{rgb}{0.58,0,0.82}

\lstset{ 
  backgroundcolor=\color{white},   % choose the background color; you must add \usepackage{color} or \usepackage{xcolor}; should come as last argument
  basicstyle=\footnotesize,        % the size of the fonts that are used for the code
  breakatwhitespace=false,         % sets if automatic breaks should only happen at whitespace
  breaklines=true,                 % sets automatic line breaking
  captionpos=b,                    % sets the caption-position to bottom
  commentstyle=\color{mygreen},    % comment style
  deletekeywords={...},            % if you want to delete keywords from the given language
  escapeinside={\%*}{*)},          % if you want to add LaTeX within your code
  extendedchars=true,              % lets you use non-ASCII characters; for 8-bits encodings only, does not work with UTF-8
  %firstnumber=1000,                % start line enumeration with line 1000
  %frame=single,                    % adds a frame around the code
  keepspaces=true,                 % keeps spaces in text, useful for keeping indentation of code (possibly needs columns=flexible)
  keywordstyle=\color{blue},       % keyword style
  language=Matlab,                 % the language of the code
  morekeywords={*,...},            % if you want to add more keywords to the set
  numbers=left,                    % where to put the line-numbers; possible values are (none, left, right)
  numbersep=1pt,                   % how far the line-numbers are from the code
  numberstyle=\tiny\color{mygray}, % the style that is used for the line-numbers
  rulecolor=\color{black},         % if not set, the frame-color may be changed on line-breaks within not-black text (e.g. comments (green here))
  showspaces=false,                % show spaces everywhere adding particular underscores; it overrides 'showstringspaces'
  showstringspaces=false,          % underline spaces within strings only
  showtabs=false,                  % show tabs within strings adding particular underscores
  %stepnumber=2,                    % the step between two line-numbers. If it's 1, each line will be numbered
  stringstyle=\color{mymauve},     % string literal style
  tabsize=2,	                   % sets default tabsize to 2 spaces
  title=\lstname                   % show the filename of files included with \lstinputlisting; also try caption instead of title
}


\geometry{left=2cm,right=2cm,top=3cm,bottom=2cm}

\begin{document}
\renewcommand{\labelitemi}{$\bullet$}
\begin{center}
\large{{\bf Introducci\'on a la Programaci\'on y An\'alisis Num\'erico\\[5pt] A\~no 2021}}\\[5pt]
\large{{\bf Pr\'actica 8: Soluci\'on de Ecuaciones No-lineales.}}\\
\rule{17cm}{1pt}
\bigskip
\end{center}

En esta pr\'actica estudiaremos como resolver num\'ericamente el problema de la b\'usqueda de ra\'ices. Las ecuaciones no lineales en general no pueden ser resueltas usando m\'etodos anal\'iticos, siendo necesario recurrir a soluciones aproximadas basadas en m\'etodos iterativos. Luego, a partir de una aproximaci\'on inicial, se genera una sucesi\'on de aproximaciones donde el proceso iterativo se contin\'ua hasta que la aproximaci\'on se encuentra pr\'oxima a la ra\'iz $\alpha$ dentro de una tolerancia $\epsilon > 0$ preestablecida. Como la ra\'iz no es conocida, no podemos calcular el error absoluto entre la aproximaci\'on y la ra\'iz, $|x_{n+1} -\alpha|$. En su lugar, en general el criterio para detener las iteraciones ser\'a $|x_{n+1} - x_{n}| < \epsilon|x_{n+1}|$. Esta desigualdad controla en cierta medida el error relativo en cada paso.\par
Puede ocurrir que nunca se llegue a la ra\'iz con la tolerancia buscada, ya sea porque el m\'etodo diverge o porque la tolerancia preestablecida no es razonable. Luego, para evitar este problema es conveniente considerar un n\'umero m\'aximo de iteraciones a realizarse. De alcanzarse este valor, el programa se detendr\'a y el problema tendr\'a que ser analizado con m\'as cuidado. \par
Para la resoluci\'on de la pr\'actica ser\'a de mucha utilidad graficar la funci\'on para poder aproximar los valores iniciales correspondientes a cada m\'etodo.\par


\subsection*{M\'etodo de Bisecci\'on}

El m\'etodo de Bisecci\'on consiste en dividir a la mitad el intervalo donde se encuentra la ra\'iz y en cuyos extremos la funci\'on cambia de signo, hasta que dos iteraciones sucesivas sean menor a la tolerancia preestablecida. Es decir, a partir de un intervalo [$a, b$] en el que se encuentra la ra\'iz, con f(a) y f(b) de signos diferentes, se computa el punto medio del intervalo, $x_{0}= (a+b)/2$, y se analiza en cual de los dos subintervalos [$a, x_{0}$] o [$x_{0}, b$] se encuentra la ra\'iz analizando el signo de la funci\'on en el nuevo extremo, $x_{0}$. El procedimiento se sigue hasta que, por ejemplo, el error absoluto entre dos iteraciones sucesivas sea menor que la tolerancia, $|x_{n+1} - x_{n}| < \epsilon $. La convergencia del m\'etodo es lineal. 

\begin{itemize}

\item[{\bf Ej. 1:}] Describa en qu\'e consiste el m\'etodo de \textit{bisecci\'on}. ?`Qu\'e tipo de problemas nos permite resolver? ?`En qu\'e casos el algoritmo conducir\'ia a un resultado err\'oneo?

\item[{\bf Ej. 2:}] Se sabe que la funci\'on $f(t)=te^{-t}+t^{2}-1$ tiene una ra\'iz en el intervalo $[0,1]$. 

\item[{\bf a)}] Utilizando el m\'etodo de bisecci\'on, ?`Cu\'antas iteraciones del algoritmo le asegurar\'ian encontrar dicha ra\'iz con una exactitud mayor o igual a $10^{-4}$?

\item[{\bf b)}] Escriba una {\it funci\'on} de OCTAVE/MATLAB que encuentre dicha ra\'iz utilizando como criterio de corte la estimaci\'on num\'erica del error absoluto $\vert p_{n+1}-p_{n}\vert < 10^{-4}$. ?`cu\'antas iteraciones del algoritmo se realizaron para llegar a la aproximaci\'on requerida? Compare con lo obtenido en el inciso anterior.

%\item[{\bf c)}] Modificar el programa del  Ej. 2. b) para que indique cu\'antas iteraciones del algoritmo se realizaron para llegar a la aproximaci\'on requerida. Comparar con los resultados obtenidos anteriormente.

\item[{\bf Ej. 3:}] Encontrar, en forma aproximada, los dos primeros puntos de intersecci\'on de las gr\'aficas de las funciones $f(x) = 4 + cos(x+1)$ y $g(x)=e^x sen(x)$, correspondientes al primer cuadrante. Elija como criterio de corte un error relativo menor al $1\%$.
\underline{\textit{Sugerencia}}: Grafique ambas funciones para saber por d\'onde empezar a buscar.

%\item[{\bf Ej. 4:}] Determinar, con un error relativo menor a $10^{-4}$, la ra\'iz de la funci\'on $x^3 + 4x^2 -10$ en el intervalo $\left[1,2\right]$. Generar una tabla con los valores de $p_n$ y $f(p_n)$ para cada iteraci\'on $n$. En funci\'on a lo observado, ?`qu\'e otro criterio de corte podr\'ia utilizarse?

%\item[{\bf Ej. 5:}] Si un c\'irculo de radio \textit{a} rueda en el plano a lo largo del eje horizontal, un punto $P$ de la circunferencia trazar\'a una curva denominada cicloide. Esta curva puede expresarse mediante las siguientes ecuaciones param\'etricas: $x(t) = a(t - \sin t)$ e $y(t) = a(1 -\cos t)$. Suponga que el radio, \textit{a}, es de $1$ metro, si $(x, y)$ se miden en metros y $t$ representa tiempo en segundos, determine el primer instante en el que la magnitud de la velocidad es $0.43 m{/}s$. Use el m\'etodo de biseci\'on, y estime el instante con una precisi\'on de $E=10^{-4}$.

%\item[{\bf Ej. 6:}] En una planta de abastecimiento de combustible se tiene un tanque de forma esf\'erica. El volumen del l\'iquido almacenado en el tanque se puede calcular mediante la siguiente f\'ormula: $v(h) = \pi h^2 (R-h{/} 3 )$, donde $h$ representa la altura del l\'iquido dentro del tanque medida desde la parte inferior del tanque y $R$ su radio $(0 \leq h \leq 2R)$. Suponga que el tanque tiene un radio de $2 m$. Calcule la altura que debe tener el l\'iquido para que el tanque contenga $28 m^3$ . Calcule el resultado con una tolerancia $E=10^{-4}$ .
\end{itemize}

\subsection*{ M\'etodo de Punto Fijo}

Para el m\'etodo de Punto Fijo es necesario reescribir la ecuaci\'on $f(x) = 0$ en la forma $x = g(x)$. Luego, a partir de una aproximaci\'on inicial $x_{0}$, se obtiene la sucesi\'on de aproximaciones $x_{1}, x_{2}, ... $ mediante:
\begin{equation*} 
\centering
x_{n+1} = g(x_{n}), \qquad n= 1, 2, ...  
\end{equation*}
El procedimiento se sigue hasta que $|x_{n+1} - x_{n}| < \epsilon|x_{n+1}|$. \par
Propiedad de convergencia: sea $g(x)$ continua y derivable en el intervalo $I$ que contiene a la ra\'iz $\alpha$. Si existe una constante positiva $0<k<1$ tal que $|g'(x)| \leq k$ para todo $x \in I$, entonces el esquema de iteraci\'on converge a la ra\'iz $\alpha$ si la aproximaci\'on inicial est\'a lo suficientemente cerca del punto fijo. Se tiene as\'i un criterio \'util a la hora de decidir cu\'al f\'ormula iterativa puede y cu\'al conviene que sea utilizada. Por otra parte, puede demostrarse que si la ra\'iz es simple, la convergencia del m\'etodo es lineal.

\begin{itemize}
\item[{\bf Ej. 4:}] Describa en qu\'e consiste el m\'etodo de \textit{punto fijo}. ?`Bajo que condiciones se asegura la convergencia del m\'etodo?

\item[{\bf Ej. 5:}] 

\item [{\bf a)}] La ecuaci\'on $x + \mathrm{log}(x) = 0$ tiene una ra\'iz cuyo valor aproximado es $\alpha \approx 0.5$. ?`De las siguientes f\'ormulas iterativas, cu\'ales de ellas pueden usarse y c\'ual deber\'ia usarse?
  
%   \begin{enumerate}[i)]
     i) $x_{n+1} = -\mathrm{log}(x_{n})$\\
    ii) $x_{n+1} = e^{-x_{n}}$\\
   iii) $x_{n+1} = \frac{x_{n} + e^{-x_{n}}}{2}$
%   \end{enumerate}

\item [{\bf b)}] Implemente el c\'odigo de punto fijo a partir de una funci\'on y verifique lo determinado en el inciso anterior. ?`Qu\'e tolerancia deber\'ia utilizarse si se quiere tener cinco d\'igitos significativos? 

\item[{\bf Ej. 6:}] 
\item[{\bf a)}] Grafique la funci\'on $f(x)= e^x-\pi x$ en el intervalo $[0,2]$. ?`Cu\'antas ra\'ices observa?

\item[{\bf b)}] Utilice la equaci\'on $x=g(x)=e^x{/}\pi$ para encontrar la ra\'iz de $f(x)$ m\'as pr\'oxima a $0$. \underline{\textit{Ayuda}}: Elija un valor inicial cercano la ra\'iz. Grafique las funciones $g(x)$ e $y=x$ en dicho intervalo.

\item[{\bf c)}] Utilice la misma ecuaci\'on para encontrar la segunda ra\'iz? ?`Qu\'e observa? ?`Qu\'e puede concluir de este \'ultimo inciso?

\item[{\bf d)}] Encuentre el intervalo de convergencia para el problema planteado, es decir, hallar los valores de $x$, tal que $|g'(x)|<1$. ?`Qu\'e puede decir de $g(x)$ en este caso?

\item[{\bf e)}] Utilice ahora la ecuaci\'on equivalente $x=g(x)=ln(\pi x)$ para encontrar la segunda ra\'iz de $f(x)$. Comparar con los resultados del inciso c). 

%\item[{\bf Ej. 10:}] Una farmac\'eutica produce $q$ gramos de cierto producto qu\'imico al d\'ia. El costo de producir esta cantidad de producto esta dado por:
%\begin{equation}
% C(q)=1000+2q+3q^{2/3}.
%\end{equation} 
%La firma puede vender este producto a $\$4$ el gramo. ?`Cu\'antos gramos de este producto deber\'ia vender la compa\~n\'ia diariamente para poder tener ganancia?.
\end{itemize}

\subsection*{M\'etodo de Newton-Raphson}

Este m\'etodo aproxima a la ra\'iz buscada a partir de rectas tangentes a la funci\'on f(x). El m\'etodo comienza a partir de una aproximaci\'on inicial $x_{0}$ cercana a la ra\'iz, a partir de la cual se genera la sucesi\'on de aproximaciones $x_{1}, x_{2}, ...$, siendo $x_{n+1}$ la abscisa del punto de intersecci\'on del eje $x$ con la recta tangente a $f(x)$ que pasa por el punto $(x_{n}, f(x_{n}))$. La f\'ormula de iteraci\'on resulta:
\begin{equation*}
\centering
x_{n+1} = x_{n} -\frac{f(x_{n})}{f'(x_{n})}, \qquad  n= 1, 2, ...
\end{equation*} 
donde el criterio de corte estar\'a dado por: $|x_{n+1} - x_{n}| < \epsilon|x_{n+1}|$.\par
La convergencia del m\'etodo depender\'a de la multiplicidad de la ra\'iz. El m\'etodo tendr\'a convergencia cuadr\'atica si la ra\'iz es simple, es decir, si $f'(\alpha) \neq 0$. En caso de que la ra\'iz sea de multiplicidad $m$, la convergencia ser\'a lineal.

\begin{itemize}
\item[{\bf Ej. 7:}] Describa en qu\'e consiste el m\'etodo de \textit{Newton-Raphson}. ?`C\'omo puede estimar el error para un n\'umero $n$ de iteraciones?

\item[{\bf Ej. 8:}] 
\item [{\bf a)}] Implemente el c\'odigo de Newton como una funci\'on.
\item [{\bf b)}] Calcule las raices de las siguientes funciones. ?`Qu\'e cantidad de iteraciones ser\'an necesarias si se requiere ocho d\'igitos significativos? Compare los resultados de la primera funci\'on con los correspondientes al ejercicio 2.\\

	i) $xe^{-x} + x^{2} -1$\\
	ii) $\mathrm{cos}(x) -x = 0$   \\
    iii) $e^{x} -x -1 = 0$

\item[{\bf c)}] De los resultados del inciso anterior se observa que para el c\'alculo de la \'ultima funci\'on dada se necesitaron m\'as iteraciones para encontrar la soluci\'on, esto se debe a que la ra\'iz es m\'ultiple. ?`Qu\'e consideraci\'on se podr\'ia hacer para obtener una funci\'on $u(x)$ que tenga las mismas raices que $f(x)$ pero simples para recuperar la convergencia cuadr\'atica? Verifique.  
\item[{\bf d}] ?`Puede aplicar el m\'etodo de bisecci\'on para calcular la ra\'iz de la \'ultima funci\'on? ?`Qu\'e ocurre?

%\item[{\bf a)}] La funci\'on \textit{newton}, detallada debajo, implementa el m\'etodo hom\'onimo. Comente el mismo indicando que operaci\'on se realiza en cada una de sus l\'ineas y que significan cada uno de sus par\'ametros.
%\begin{lstlisting}[language=matlab]
% function [x,fx,xx] = newton(f,df,x0,TolX,MaxIter)
%   % newton.m: resuelve f(x) = 0 con Newton method.
%   h      = 1e-4;
%   h2     = 2*h;
%   TolFun = eps;
%   xx(1) = x0;
%   fx    = feval(f,x0);
%   for k = 1: MaxIter
      
%     dfdx = feval(df,xx(k)); 
     % dfdx = (feval(f,xx(k) + h)-feval(f,xx(k) - h))/h2; 
%     dx = -fx/dfdx;
%     xx(k+1) = xx(k)+dx;
%     fx = feval(f,xx(k + 1));
%     if abs(fx)<TolFun | abs(dx) < TolX
%        break;
%     endif
%   endfor
%   x = xx(k + 1);
% endfunction
%\end{lstlisting}
%\item[{\bf b)}] Calcule una ra\'iz real de $f(x) = e^x - \pi x = 0$ con la f\'ormula de Newton. Utilizando el mismo valor inicial $x_0$ del Ej. $2.~c)$, ?`cu\'antas iteraciones se necesitaron para converger a la ra\'iz? ?`Qu\'e puede decir de la velocidad de convergencia de este m\'etodo?

%\item[{\bf Ej. 13:}] Utilice el m\'etodo de Newton-Raphson para encontrar el valor de $\sqrt{10}$ con un error absoluto menor a $10^{-8}$. Use como aproximaci\'on inicial de la soluci\'on al valor $x_0=3.$

%\item[{\bf Ej. 14:}] Un pr\'estamo de $X$ pesos es devuelto en $n$ cuotas de igual valor $M$, comenzando a partir del siguiente mes en el que se efectu\'o el pr\'estamo. Se puede demostrar que si la tasa de inter\'es mensual es $r$, entonces
%\begin{equation}
% Xr=M(1-\frac{1}{(1+r)^n}).
%\end{equation}
%Suponga que se pidi\'o un pr\'estamo de $\$70000$ pesos para comprar un auto y fue devuelto en 60 cuotas de %$\$1666.6$. Utilice el m\'etodo de Newton-Raphson para hallar la tasa de inter\'es mensual con cuatro cifras significativas.

%\item[{\bf Ej. 15:}] Suponga que se quiere lanzar un sat\'elite de modo que los vientos solares siempre pasen a trav\'es de dicho sat\'elite. Para esto, el mismo deber\'ia estar posicionado  siempre a lo largo de la l\'inea Tierra-Sol. Se sabe que para que esto ocurra, la distancia, $r$, que este sat\'elite debe tener respecto de la tierra debe satisfacer la siguiente equaci\'on:
%\begin{equation}
% r^3(R-r)^2\omega^2-GM_s(R-r)^2+GM_tr^2 = 0.
%\end{equation}Encuentre el valor de esta distancia suponiendo que $G=6.67 \times 10^{11}$, la masa del sol, $M_s=1.98\times10^{30}$, la masa de la tierra, $M_t=5.98\times10^{24}$[kg], la distancia Tierra-Sol, $R=1.49 \times 10^{11}$[m] y el per\'iodo de translaci\'on medio de la tierra dado $T=3.15576\times10^7$[s].

\end{itemize}

%\subsection*{Sistemas de Ecuaciones no - lineales}
%El m\'etodo de Newton-Raphson se extiende como sigue para aproximar las raices de un sistema $n$ de ecuaciones no lineales. Sea:
%
%\begin{equation*}
%\overline{F}(\overline{x}) = \overline{0} 
%\end{equation*}
%
%donde:
%
%\begin{align}
%        \begin{matrix} \overline{F}(\overline{x}) =
%          \begin{bmatrix}
%           f_{1}(x_{1}, x_{2}, ... , x_{n}) \\  
%           f_{2}(x_{1}, x_{2}, ... , x_{n})\\   
%           \vdots \\
%         f_{n}(x_{1}, x_{2}, ... , x_{n}) \\
%          \end{bmatrix} y \quad \overline{x} =
%          \begin{bmatrix}
%           x_{1} \\
%           x_{2} \\
%           \vdots \\
%           x_{n}
%         \end{bmatrix}
%    \end{matrix}
%    \nonumber
%  \end{align}
%
%A partir del desarrollo de Taylor en varias variables centrado en $\overline{x}_{0}$ (aproximaci\'on inicial) a primer orden, se tiene:
%
%\begin{equation*}
%\overline{F}(\overline{x}) \approx \overline{F}(\overline{x}_{0}) + (\overline{x} - \overline{x}_{0})J(\overline{x}_{0})
%\end{equation*}
%con $J(\overline{x}_{0}) = \frac{\partial f_{i}}{\partial x_{j}}(\overline{x}_{0})$ matriz jacobiana. Luego, despejando $\overline{x}$ (y considerando $\overline{F}(\overline{x}) = \overline{0}$), la f\'ormula de iteraci\'on para un sistema resulta:
%
%\begin{equation*}
%\overline{x}_{n+1} = \overline{x}_{n} - J^{-1}(\overline{x}_{n})\overline{F}(\overline{x}_{n}), \qquad  n= 1, 2, ...
%\end{equation*} 
%
%\begin{itemize}
%\item[{\bf Ej. 9:}] Dado un sistema de dos ecuaciones con dos inc\'ognitas, desarrolle un c\'odigo que utilice el m\'etodo de Newton-Raphson para resolver dicho sistema con una tolerancia a definir por el usuario.
%
%\item[{\bf Ej. 10:}] Resuelva los siguientes sistemas de ecuaciones:
%\begin{eqnarray*}
%\begin{cases}
%x_1-x_2+1 = 0\\
%x_1^2+ x_2^2-4 = 0
%\end{cases}
%\end{eqnarray*}
%\begin{eqnarray*}
%\begin{cases}
%3x^2 + 4y^2 = 3\\
%x^2 + y^2 = \sqrt{3{/}2}
%\end{cases}
%\end{eqnarray*}
%% Ejemplo Chapra: valores iniciales: x$_1$ = 1.5 y x$_2$ = 3.5
%\begin{eqnarray*}
%\begin{cases}
%x_1^2-x_1x_2-10 = 0\\
%x_2+ 3x_1x_2^2-57 = 0
%\end{cases}
%\end{eqnarray*}
%\begin{eqnarray*}

%En cada caso, verifique gr\'aficamente como se aproxima la soluci\'on num\'erica a la verdadera en cada iteraci\'on. Para esto, puede graficar el valor de $f(u_n,v_n)$ en funci\'on del n\'umero de iteraci\'on $n$ correspondiente, junto con el valor del t\'ermino independiente.
%\end{itemize}

\end{document}
