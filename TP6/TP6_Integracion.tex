\documentclass[10pt,a4paper,twoside,spanish]{article}	%Indica la clase de documento

\pagestyle{plain}
\usepackage{latexsym}
\usepackage[spanish]{babel} 				%Division de silabas en espa�ol.
\usepackage{ucs}

\usepackage[latin1]{inputenc} 				%Para poner acentos directamente
\usepackage[T1]{fontenc}

\usepackage[dvips]{graphicx}   				 %Insertar figuras .eps
\DeclareGraphicsExtensions{.bmp,.png,.pdf,.jpg}

\usepackage{amsmath,amssymb,amsfonts,latexsym,cancel}  %Comandos especiales
\newcommand{\sen}{\mathop{\rm sen}\nolimits}            %seno
\newcommand{\arcsen}{\mathop{\rm arcsen}\nolimits}
\newcommand{\arcsec}{\mathop{\rm arcsec}\nolimits}
\def\max{\mathop{\mbox{\rm m\?ax}}}			 %mx
\def\min{\mathop{\mbox{\rm m\?{\i}n}}}			 %mn
\usepackage[table]{xcolor} 				%Para poner color en las tablas
\usepackage{float}
\setlength\unitlength{1mm}

%*************** Agregados de formato********************
%\usepackage{shortlst}					% Intems cortos
\usepackage{multicol}					% Permite dividir en columnas
\usepackage{wrapfig}					% Texto al lado de las figuras
\usepackage[rftl]{floatflt}
\usepackage{rotating}					% Rotar elementos
\usepackage{subfig}					% Subfiguras
\usepackage{anysize}					% Configurar margenes
\marginsize{2 cm}{2 cm}{1 cm}{2 cm}
\usepackage{multirow}
\usepackage[a4paper]{geometry}
%************************************************************

%************************************************************
% Define un comando para formatear y contar los ejercicios

\newcounter{contador}			% Se crea un contador
\setcounter{contador}{1}
\newcommand{\ej}{\subsection*{Ejercicio \thecontador} \stepcounter{contador} }


\geometry{left=2cm,right=2cm,top=2cm,bottom=2cm}

\begin{document}
\begin{center}
\large{{\bf Introducci\'on a la Programaci\'on y An\'alisis Num\'erico\\[5pt] A\~no 2022}}\\[5pt]
\large{{\bf Pr\'actica 6: Integraci\'on Num\'erica}}\\
\rule{17cm}{1pt}

\bigskip
\end{center}

\section*{F\'ormulas de integraci\'on num\'erica.}
Definiremos a continuaci\'on las reglas de integraci\'on que veremos en este curso. Para las mismas necesitamos que: 

\begin{itemize}
\item $f$ sea una funci\'on suave sobre el intervalo cerrado $[a,b]$.
\item El intervalo $[a,b]$ est\'a subdividido en $n$ subintervalos de longitud $h = (b-a)/n$.
\item Se tendr\'an $n+1$ puntos equiespaciados $x_{i} = x_{0}+ih$ para $i=0, 1, ..., n$, donde $x_{0} = a$ y $x_{n} = b$. 
\end{itemize}
Luego, las reglas de integraci\'on con sus respectivos t\'erminos de error (siendo $\xi$ un punto del intervalo $(a,b)$) son:

\subsection*{Regla del rect\'angulo}
\begin{equation}
\int_{a}^{b} f(x) dx = h \sum_{i=1}^{n}f(x_{i-1}) + \frac{h(b-a)}{2}f'(\xi). \nonumber
\end{equation}

\subsection*{Regla del trapecio simple}

\begin{equation}
\int_{a}^{b} f(x) dx = (b-a) \frac{ f(x_{0}) + f(x_{n})}{2} - \frac{(b-a)^{3}}{12} f''(\xi). \nonumber
\end{equation}

\subsection*{Regla del trapecio compuesta}

\begin{equation}
\int_{a}^{b} f(x) dx = \frac{h}{2} \big ( f(x_{0}) + f(x_{n}) + 2\sum_{i=1}^{n-1}f(x_{i}) \big ) - \frac{h^{2}(b-a)}{12} f''(\xi). \nonumber
\end{equation}

\subsection*{Regla de Simpson 1/3 simple}

\begin{equation}
\int_{a}^{b} f(x) dx = \frac{(b-a)}{6} \big ( f(x_{0} + 4f((a+b)/2)) +f(x_{n})\big )  - \frac{h^{5}}{90}f^{(4)}(\xi). \nonumber
\end{equation}

\subsection*{Regla de Simpson 1/3 compuesta}

\begin{equation}
\int_{a}^{b} f(x) dx = \frac{h}{3} \big ( f(x_{0}) + f(x_{n}) + 2\sum_{i=1}^{(n/2)-1}f(x_{2i})+4\sum_{i=1}^{n/2}f(x_{2i-1}) \big) - \frac{h^{4}(b-a)}{180}f^{(4)}(\xi). \nonumber
\end{equation}

\subsection*{Cuadratura de Gauss-Legendre}
Si $x_{1}, x_{2}, .... x_{n}$ son las $n$ ra\'ices del polinom\'io de Legendre de grado $n$, la f\'ormula de cuadratura de Gauss-Legendre de $n$ puntos para estimar la integral de una funci\'on $f$ sobre el intervalo $[-1,1]$ est\'a dada por:
\begin{equation}
\int_{-1}^{1} f(x)dx \approx \sum_{1}^{n} A_{i}f(x_{i}),\quad \textnormal{donde:} \quad A_{i} = \int_{-1}^{1} \prod_{j=1}^{n} \frac{(x-x_{j})}{(x_{i}-x_{j})}dx. \nonumber
\end{equation}

El error de truncamiento est\'a dado por $E(f)=c_{n}f^{(2n)}(\xi)$, donde $\xi$ es un punto del intervalo $(-1,1)$ y $c_{n}$ es una constante. Esta f\'ormula de integraci\'on es exacta para todo polinomio de grado $\leq 2n -1$. \par
Como se puede observar, el c\'alculo de la integral dada por las f\'ormulas de cuadratura de Gauss-Legendre es en el intervalo de integraci\'on $(-1,1)$. Luego, si $b > a$, el cambio de variable $t=\frac{2x-(b+a)}{(b-a)}$ permite evaluar la integral en el intervalo $(a,b)$:
\begin{equation}
\int_{a}^{b}f(x)dx = \int_{-1}^{1} \phi (t) dt, \quad \textnormal{donde:} \quad \phi (t) = \frac{1}{2}(b-a)f \big ( \frac{1}{2}(b-a)t + \frac{1}{2}(b+a) \big ). \nonumber
\end{equation}    
\vspace{1 cm}


%%%%%%%%%%%%%%%%%%%%%%%%%%%%%%%%%%%%%%%%%%%%%%
%%%%%%%%%%%%%%%%%%%%%%%%%%%%%%%%%%%%%%%%%%%%%%
\ej Implemente mediante funciones las reglas del rect\'angulo, del trapecio y de Simpson arriba definidas. Dadas las siguientes integrales:
\begin{multicols}{2}
 $$1) \; \;\int_0^1 {\frac{1}{1+x^2}}dx$$
 
 $$2) \;\; \int_1^2{\frac{2x+1}{x^2+x}}dx$$
\end{multicols}
\begin{itemize}
\item[a)] Aproximar su valor empleando la regla trapezoidal simple. Acote el error cometido.
\item[b)] Aproximar su valor empleando la regla de Simpson simple. Acote el error cometido. 
\item[c)] Grafique y compare las aproximaciones realizadas en los incisos anteriores. Puede utilizar el siguiente c�digo como referencia en un software de c�lculo num�rico.
 
\begin{verbatim}
a=0;      % limite inferior de integraci�n
b=1;      % limite superior de integraci�n
f=@(x)(1./(1+x.^2));      % define la funcion a integrar

xt=a:0.01:b;        % define un vector para graficar
yt=f(xt);
plot(xt,yt,'Linewidth',2)     % grafica la funcion definida

P1=f(a)+(f(b)-f(a))/(b-a).*(xt-a); 
hold on 
plot(xt,P1,'r')     % aproximaci�n mediante un trapecio

Itrap=(f(b)+f(a))/2*(b-a) % calcula el valor de la integral

h=(b-a)/2;          % aproximaci�n con un polinomio de orden 2 
x2=[a (b+a)/2 b];
y2=f(x2);
plot(x2,y2,'o')

P2= f(a)+(f(a+h)-f(a))/h.*(xt-a)+ (f(b)-2*f(a+h) +f(a))/2/h.^2.*(xt-a).*(xt-a-h); 
plot(xt,P2,'g')

Isimp=h/3*(f(b)+4*f(a+h)+f(a)) % Calcula el valor de la integral 
\end{verbatim} 
 
\item[d)] Aproximar su valor empleando la regla trapezoidal compuesta utilizando un paso $h=0,1$.
\item[e)] Aproximar su valor empleando la regla de Simpson compuesta utilizando un paso $h=0,1$. Comparar con el inciso anterior.

\item[f)] Obtener el valor exacto de cada integral y calcular el valor del error porcentual de cada resultado obtenido.

\item[g)] Realice un gr�fico esquem�tico comparando las reglas simple y compuesta de cada m�todo de calculo. Prediga como evoluciona el valor de la integral al aumentar el paso $h$.

\end{itemize} 

%%%%%%%%%%%%%%%%%%%%%%%%%%%%%%%%%%%%%%%%%%%%%%
%%%%%%%%%%%%%%%%%%%%%%%%%%%%%%%%%%%%%%%%%%%%%%
\ej Aplicar la reglas del rect\'angulo, trapecio compuesta y Simpson compuesta a la integraci\'on de $\sqrt{x}$ entre los
argumentos 1,00 y 1,30, utilizando la siguiente tabla:


\begin{table}[H]
\begin{tabular}{|c|c|c|c|c|c|c|c|}
 \hline
$x_i$ & 1,00 & 1,05 & 1,10 & 1,15 & 1,20 & 1,25 & 1,30 \\ 
\hline
$f(x_i)$ & 1,00000 & 1,02470 & 1,04881  & 1,07238 &  1,09544 & 1,11803 & 1,14017 \\
\hline
\end{tabular} 
\centering
\end{table}

\noindent Comparar los resulados de los distintos m\'etodos con el valor anal�tico de la integral.
% Acotar el error de discretizaci\'on. 

%%%%%%%%%%%%%%%%%%%%%%%%%%%%%%%%%%%%%%%%%%%%%%

\ej Sabiendo que la integral de $\int_{0}^{\pi} sen(x) dx$ = 2, determine c\'ual es la menor cantidad de puntos necesarios considerar en la integraci\'on n\'umerica por las regla de trapecios y Simpson compuestas, respectivamente, para garantizar un error absoluto menor a 0,00002. Para ello, utilice la expresi�n del error de cada m�todo. ?`A qu\'e se debe la diferencia en los resultados obtenidos? Escriba tanto la cantidad de puntos determinados como el error absoluto obtenido en cada caso.


%%%%%%%%%%%%%%%%%%%%%%%%%%%%%%%%%%%%%%%%%%%%%%
\ej Calcular la integral de los siguientes datos tabulados empleando el m�todo de trapecios y Simpson.   

\begin{table}[H]

\begin{tabular}{|c|c|c|c|c|c|c|}
\hline 
Puntos & 0 & 1 & 2 & 3 & 4 & 5 \\ 
\hline 
 $x$ & 0 & 0,1 & 0,2 & 0,3 & 0,4 & 0,5 \\ 
\hline 
 $f(x)$ & 1 & 7 & 4 & 3 & 5 & 2 \\ 
\hline 
\end{tabular} 
\centering
\end{table}

�Es posible emplear las f�rmulas de cuadraturas Gaussianas para el calculo de la integral con los datos suministrados?

%%%%%%%%%%%%%%%%%%%%%%%%%%%%%%%%%%%%%%%%%%%%%%
%%%%%%%%%%%%%%%%%%%%%%%%%%%%%%%%%%%%%%%%%%%%%%
%\ej Calcular por la f\'ormula de los trapecios, con $h=0.25$ y $h=0.125$
%$$
%\int_0^3  \frac{1}{1+x^3}
%$$
%Con las soluciones calculadas �es posible obtener una mejor aproximaci�n del valor de la integral?

%%%%%%%%%%%%%%%%%%%%%%%%%%%%%%%%%%%%%%%%%%%%%%
%%%%%%%%%%%%%%%%%%%%%%%%%%%%%%%%%%%%%%%%%%%%%%
\ej Sabiendo que:
$$
ln(x)=\int_1^x \frac1{t} \,dt
$$
aproximar $ln(x)$ para x=1,1 con h=0,1 empleando la f\'ormula trapezoidal. Acotar el error y comparar con el error real cometido.


%%%%%%%%%%%%%%%%%%%%%%%%%%%%%%%%%%%%%%%%%%%%%%
%%%%%%%%%%%%%%%%%%%%%%%%%%%%%%%%%%%%%%%%%%%%%%
%\ej Calcular el �rea de forma exacta de las regiones limitadas por las siguientes funciones empleando el m�todo num�rico �ptimo.  
%\begin{multicols}{2}
%
% $$ A= \begin{cases} y=x+1 \\ y=0 \\ x=0 \\ x=2 \end{cases} $$
%
% $$B=\begin{cases} y= x^2+1 \\ y= 0 \\ x=1 \\ x= 2 \end{cases}$$
%\end{multicols}

%%%%%%%%%%%%%%%%%%%%%%%%%%%%%%%%%%%%%%%%%%%%%%
%%%%%%%%%%%%%%%%%%%%%%%%%%%%%%%%%%%%%%%%%%%%%%
\ej Eval�e las siguientes integrales, utilizando las f�rmulas de cuadratura de Gauss-Legendre. Con dos, tres y cuatro puntos. (Obtener los pesos y ra�ces de una tabla normalizada)

\begin{itemize}
\begin{multicols}{2}
\item[a)]$\displaystyle \int_0^3 \frac{e^{x}sen(x)}{1+x^2} \, dx$

\item[b)]$\displaystyle \int_0^1 \frac{sen(x)}{x} \,dx$

\item[c)]$\displaystyle \int_{-1}^1 \frac{arctg(x)}{1+x} \,dx$ 

\item[d)]$\displaystyle \int_0^1 \frac{ln(1+x)}{1+x^2} \,dx$
\end{multicols}
\end{itemize}

%%%%%%%%%%%%%%%%%%%%%%%%%%%%%%%%%%%%%%%%%%%%%%
%%%%%%%%%%%%%%%%%%%%%%%%%%%%%%%%%%%%%%%%%%%%%%
%\ej Hallar num\'ericamente el volumen del s\'olido de revoluci\'on, obtenido
%cuando la curva $y=x e^{-x} cos(x)$ gira alrededor del eje $(0x)$ entre
%$x=0$ \ y \ $x=\pi$.
%
%%%%%%%%%%%%%%%%%%%%%%%%%%%%%%%%%%%%%%%%%%%%%%%
%%%%%%%%%%%%%%%%%%%%%%%%%%%%%%%%%%%%%%%%%%%%%%%
%\ej Dada la secci\'on transversal del canal de la figura y los valores del
%ancho $W$ medidos para distintas profundidades $h$, calcular el \'area de la
%misma aplicando: a) Regla del Trapecio. b) F\'ormula de Simpson.
%
%
%\begin{minipage}{0.25 \textwidth}
%\begin{table}[H]
%\begin{tabular}{c|c}
%h [m] & W [m] \\ 
%\hline 
%0 & 0 \\ 
%\hline 
%1 & 2 \\ 
%\hline 
%2 & 3 \\ 
%\hline 
%3 & 6 \\ 
%\end{tabular} 
%\centering
%\end{table}
%\end{minipage}
%\hfill \begin{minipage}{0.75 \textwidth}
%\begin{figure}[H]
%\includegraphics[scale=2]{canal.pdf}
%\centering
%\end{figure}
%
%\end{minipage}




%%%%%%%%%%%%%%%%%%%%%%%%%%%%%%%%%%%%%%%%%%%%%%
%%%%%%%%%%%%%%%%%%%%%%%%%%%%%%%%%%%%%%%%%%%%%%
%\ej Calcular empleando un software de c�lculo num�rico la siguiente integral
%$$ I= \int_{-2}^2 e^{\frac{-x^2}{2}} \,dx $$
%cuyo valor exacto es $I=2.3925$. Resolver por regla trapezoidal, de Simpson y
%de cuadratura de Gauss con $h=1.0$, $h=0.5$ y $h=0.25$. Realizar un cuadro
%comparativo con los resultados obtenidos.


%%%%%%%%%%%%%%%%%%%%%%%%%%%%%%%%%%%%%%%%%%%%%%
%%%%%%%%%%%%%%%%%%%%%%%%%%%%%%%%%%%%%%%%%%%%%%
%\ej �Qu� diferencias puede mencionar entre los m�todos de integraci�n de Newton-Cotes y Gauss?
%
%
%
%%%%%%%%%%%%%%%%%%%%%%%%%%%%%%%%%%%%%%%%%%%%%%%
%%%%%%%%%%%%%%%%%%%%%%%%%%%%%%%%%%%%%%%%%%%%%%%
%\ej La velocidad hacia arriba de un cohete se puede calcular con la siguiente f�rmula:
%
%$$v= u \cdot ln \left( \frac{M_0}{M_0 - q \, t}\right) -g \, t $$
%donde: 
%
% $u :$ velocidad a la cual se expulsa el combustible relativa al cohete.
%
% $M_0 :$ masa inicial del cohete.
% 
% $q :$ raz�n de consumo de combustible.
% 
% $g:$ aceleraci�n debido a la gravedad. Se supone constante.
% 
%\noindent Si $u = 2000 \; m/s$ , $M_0 = 150000 \; kg$ y $q=2600 \; kg/s$ calcule la altura del cohete luego de 30 segundos.
%
%%%%%%%%%%%%%%%%%%%%%%%%%%%%%%%%%%%%%%%%%%%%%%%
%%%%%%%%%%%%%%%%%%%%%%%%%%%%%%%%%%%%%%%%%%%%%%%
%\ej Indique si cada proposici�n es verdadera o falsa. Justifique la respuesta.
%
%\begin{itemize}
%
%\item[a)] El m�todo de integraci�n del trapecio integra exactamente un polinomio que tiene a lo sumo orden uno.
%\item[b)] El m�todo de Simpson integra exactamente un polinomio que tiene a lo sumo orden dos.
%\item[c)] El error de integrar mediante la f�rmula de Simpson utilizando un n�mero par de subintervalos es aproximadamente el doble al error de integrar con un n�mero impar de subintervalos.
%\item[d)] Para integrar una funci�n en un intervalo [a, b] es necesario conocer los valores de f(a) y f(b) por lo menos, independientemente del m�todo de integraci�n utilizado. Sin embargo esta condici�n no es suficiente
%
%\end{itemize}

\end{document}

















