\documentclass[11pt]{article}

\usepackage{amsmath,amsfonts,amsthm,amssymb}
\usepackage[latin1]{inputenc}
\usepackage[spanish]{babel}
\usepackage{graphicx}
\usepackage[a4paper]{geometry}
\usepackage{subfigure}

\geometry{left=2cm,right=2cm,top=2cm,bottom=2cm}

\begin{document}
\begin{center}
\large{{\bf Introducci\'on a la Programaci\'on y An\'alisis Num\'erico\\[5pt] A\~no 2022}}\\[5pt]
\large{{\bf Pr\'actica 4: Ajuste}}\\
\rule{17cm}{1pt}

\bigskip
\end{center}


%%%%%%%%%%%%%%%%%%%%%%%%%%%%%%%%%%%%%%%%%%%%%%
{\bf Ej. 1:} Con los datos de la siguiente tabla ajuste los valores por el m�todo de m�nimos cuadrados a
una funci�n: a) lineal, b)polin�mica de $2$ grado, c) potencial y d) exponencial. Eval�e el error cuadr�tico medio en cada caso.

\begin{table}[ht]
\begin{tabular}{|c|c|c|c|c|c|}\hline

X & 0.1 & 0.2 & 0.3 & 0.4 & 0.5 \\ \hline
Y & 2 & 2.9  & 5 & 6.7 & 12 \\ \hline
\end{tabular}
\centering
\end{table}

%%%%%%%%%%%%%%%%%%%%%%%%%%%%%%%%%%%%%%%%%%%%%%


{\bf Ej. 2:} Dada la siguiente tabla de datos obtenga las funciones de ajuste utilizando el m\'etodo de m\'inimos cuadrados, considerando polinomios de grado 1, 2 y 3. Utilizando OCTAVE/MATLAB grafique los datos de la tabla y los polinomios de ajuste.
\begin{table}[ht]
\begin{tabular}[c]{|c|c|c|c|c|c|c|}
\hline
$x$ & -1 & -0.5 & 2 & 2.25 & 4 & 5 \\
\hline
$f(x)$ & 9.1 & 6 & -3.5 & -3.3 & -0.9 & 2.5 \\
\hline
\end{tabular}
\centering
\end{table}

\textquestiondown Qu\'e ajuste le parece que representa mejor los datos? �Qu\'e par\'ametro utilizar\'ia como indicador de la calidad del ajuste? \textquestiondown Cu\'al elegir�a para manipular algebraicamente? �Tendr� sentido seguir aumentando el orden del polinomio de ajuste?


{\bf Ej. 3:}
\begin{itemize}
 \item[{\bf a)}] Suponga que se cuenta con un conjunto de $m+1$ datos. Use la funci�n \textbf{help} de OCTAVE/MATLAB para entender la forma en que opera la funci�n \textbf{polyfit} .

\item[{\bf b)}] Utilice la funci�n \textbf{polyfit}  con el conjunto de datos del \textbf{Ej. 1} para $n=1,2,3$ y compare con lo obtenido previamente. �Qu� pasa si el grado del polinomio es igual a $m$? Describa la salida de la funci\'on \textbf{polyfit}  para esos casos.

\item[{\bf c)}] Basado en el m�todo de m�nimos cuadrados, busque una transformaci�n que linealice el problema y explique c�mo realizar un ajuste de datos considerando la funci�n $f(t)= \frac{at}{b+t}$. Para corroborar, genere datos y ajuste los valores de $a$ y $b$
\end{itemize}



%%%%%%%%%%%%%%%%%%%%%%%%%%%%%%%%%%%%%%%%%%%%%%

%%%%%%%%%%%%%%%%%%%%%%%%%%%%%%%%%%%%%%%%%%%%%%
{\bf Ej. 4:} Los datos de un ensayo de una barra de acero se presentan en la tabla. Se trata de ajustar una curva representativa, por el m�todo de m�nimos cuadrados, de la relaci�n carga-alargamiento $(P-\Delta L)$. Grafique los puntos y adopte una o m�s curvas para el ajuste. Grafique posteriormente las curvas obtenidas superpuesta a los puntos. �Qu� alargamiento ocurrir� en la barra si se aplica una carga de siete toneladas?

\begin{table}[h]
\begin{tabular}{|c|c|c|c|c|c|c|c|c|c|c|c|c|c|} \hline
$\Delta L \times 10^2$ & 0 & 3 & 6 & 9 & 12 & 15 & 20 & 20 & 25 & 30 & 35 & 40 & 45 \\ \hline
P [kg] & 500 & 1900 & 3250 & 4300 & 5450 & 6600 & 8100 & 9000 & 9350 & 9500 & 9700 & 9850 & 10000 \\ \hline
\end{tabular}
\centering
\end{table}

%%%%%%%%%%%%%%%%%%%%%%%%%%%%%%%%%%%%%%%%%%%%%%

{\bf Ej. 5: Ejercicio adicional:}
\bigskip

Se mide la transferencia entre tensi�n y corriente que ocurre en un circuito el�ctrico. El conjunto de medidas se encuentra en el archivo adjunto \verb"tp2ej2.txt". �Cu\'ales son los datos obtenidos experimentalmente? �Cu\'al es el par\'ametro del circuito a estimar?
\begin{itemize}
 \item[{\bf a)}] Proponga una funci�n de ajuste acorde al comportamiento que observa de los datos.
 \item[{\bf b)}] Se pretende simplificar el comportamiento del  dispositivo no lineal para lo cual se ajustan los datos mediante un polinomio lineal. �Cu�l es el valor de resistencia equivalente?  Recuerde que la resistencia es un dispositivo pasivo.
 \item[{\bf c)}] �Puede afirmar que el error al estimar el valor de corriente para una determinada tensi\'on estar\'a por debajo de un cierto valor umbral? �Por qu\'e?
\end{itemize}
\medskip
\end{document}
