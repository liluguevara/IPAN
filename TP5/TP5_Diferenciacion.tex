\documentclass[11pt]{article}

\usepackage{amsmath,amsfonts,amsthm,amssymb,latexsym}%,stmaryrd}
%\usepackage[spanish]{babel}
\usepackage{graphicx}
\usepackage{graphics}
\usepackage{epsfig}
\usepackage{epstopdf}
\usepackage[a4paper]{geometry}
\usepackage[latin1]{inputenc}
%\usepackage{subfigure}
\usepackage{float}
%\usepackage{fancyhdr}
%\usepackage{enumitem}
\usepackage{textcomp}

\geometry{left=2cm,right=2cm,top=2cm,bottom=2cm}

%************************************************************
% Define un comando para formatear y contar los ejercicios

\newcounter{contador}			% Se crea un contador
\setcounter{contador}{1}
\newcommand{\ej}{\subsection*{Ejercicio \thecontador} \stepcounter{contador} }

%************************************************************

\begin{document}
\begin{center}
\large{{\bf Introducci\'on a la Programaci\'on y An\'alisis Num\'erico\\[5pt] A\~no 2022}}\\[5pt]
\large{{\bf Pr\'actica 5: Diferenciaci\'on Num\'erica}}\\
\rule{17cm}{1pt}

\bigskip
\end{center}

\ej La derivada de $f$, en un punto $a$, puede ser estimada a partir del cociente incremental
\[
f'(a)\simeq  \frac{f(a+h)-f(a)}{h},
\]
para valores peque\~nos de $h$. La expresi\'on se conoce con el nombre de {\it f\'ormula de diferencia progresiva} si $h>0$ o {\it regresiva} si $h<0$.
 
\begin{itemize}
\item[a)] Demostrar que el error de truncamiento de la f\'ormula de diferencia est\'a dado por:
\[  
E(f)=-\frac{h}{2}f''(\xi)
\]
para alg\'un $\xi$ entre $a$ y $a+h$.
\end{itemize}
La {\it f\'ormula de diferencias centradas} est\'a dada por:
\[
f'(a)\simeq  \frac{f(a+h)-f(a-h)}{2h}
\] 
\begin{itemize}
\item[b)] Demostrar que el error de truncamiento de la f\'ormula de diferencias centradas est\'a dado por:
\[
E(f)=-\frac{h^2}{6}f'''(\xi),
\]
donde $\xi$ se encuentra entre $a-h$ y $a+h$, constituyendo una mejor aproximaci\'on que la f\'ormula anterior. 
\end{itemize}

\ej Implemente los c\'odigos para las f\'ormulas de diferencias regresivas, progresivas y  centradas como funciones. \par

\noindent Aproximar $f'(0.1)$ para $f(x)=sen(x)$ empleando la f\'ormula de diferencia centrada con diferentes valores de $h$. 
Comenzar con $h=10$ y reducir en forma sucesiva el paso a la d\'ecima parte del paso anterior. Imprimir para cada $h$ el valor estimado de la derivada y el error cometido (al menos 25 veces). Comentar los resultados obtenidos. �A qu\'e se debe lo observado? �Cu\'al parece ser el rango del valor apropiado para $h$? 


\ej Mostrar que si los errores de redondeo por la utilizaci\'on de aritm\'etica finita en la evaluaci\'on de $f$ est\'an acotados por alg\'un $\delta > 0$ y la derivada tercera de $f$ est\'a acotada por $M > 0$, entonces
\[
\Big|f'(a)-\frac{\hat{f}(a+h)-\hat{f}(a-h)}{2h}\Big| \le\frac{\delta}{h}+\frac{h^2}{6}M,
\]
donde $\hat f$ denota la evaluaci\'on de $f$ en aritm\'etica finita. Adem\'as, demostrar que el valor {\it \'optimo} de $h$, definido como el valor de $h$ para el cual la suma de las magnitudes del error de redondeo y truncamiento se minimizan, se expresa por
\[
h_{opt}=\Bigl(\frac{3\delta}{M}\Bigr)^{1/3}.
\]

Leer y analizar el ejemplo 4 de la secci�n 4.1 (p�g 174) del libro \textbf{ An�lisis Num�rico, Burden R. y Faires D.} S�ptima edici�n.

\ej Demostrar que $f''(a)$ se puede aproximar por la expresi\'on de diferencias
\[
f''(a)\simeq \frac{f(a-h)-2f(a)+f(a+h)}{h^2}
\]
con un t\'ermino de error dado por
\[
E(f)=-\frac{h^2}{12}f^{(4)}(\xi),
\]
donde $\xi$ se encuentra entre $a-h$ y $a+h$.

%\ej A partir de una tabla de valores originados por la funci\'on $f(x)=cos(x)$ en el intervalo $[0,0.8]$, aproximar la derivada primera en  $x=0.8$ utilizando la f\'ormula de derivada hacia atr\'as. Calcular para distintos valores de $h$ y graficar el error absoluto en funci\'on de $h$.

\ej Aproximar num�ricamente la derivada segunda de $f(x)=cos(x)$ en el punto $x=0.5$ con un valor de $h=0.1$, $h=0.01$ y $h=0.001$. Calcular el error en cada caso y obtener conclusiones. �Con qu\'e valor de $h$ es conveniente aproximar la derivada para obtener un menor error?  

\ej En un laboratorio se obtuvieron los siguientes pares de valores distancia-tiempo para un objeto:

\begin{figure}[H]
\begin{tabular}{|c|c|c|c|c|c|c|}
\hline 
t(s) & 0 & 1 & 2 & 3 & 4 & 5 \\ 
\hline 
d(cm) & 0 & 2 & 8 & 18 & 32 & 50 \\ 
\hline 
\end{tabular} 
\centering
\end{figure}

\begin{itemize}
\item[a)]Mediante diferenciaci�n num�rica determine la velocidad y la aceleraci�n del objeto en cada momento.
\item[b)]Describa el orden del error cometido en cada aproximaci�n.

\end{itemize}


\ej Dado el siguiente conjunto de datos: 

\begin{table}[H]

 \begin{tabular}{|c|c|c|c|c|c|c|c|c|c|}
\hline 
$x_i$ &  0.5 &  0.75 &    1.0 &    1.25 &  1.5 &    1.75 &    2.0 &    2.25 &    2.5 \\ 
\hline 
$f(x_i)$ &  3.2974  &  2.8227  &  2.7183  &  2.7923  &  2.9878 &   3.2883  &  3.6945  &  4.2168 &   4.8730 \\ 
\hline 
\end{tabular} 
\centering
\end{table}

\begin{itemize}
\item[a)] Calcular la pendiente de la recta tangente en cada punto $x_i$ dado utilizando el esquema de diferenciaci�n m\'as adecuado.
\item[b)] Sabiendo que los puntos pertenecen a la funci�n $\displaystyle f(x)= \frac{e^x}{x}$. Calcular anal�ticamente los valores de $f'(x_i)$, el error absoluto y porcentual.   

\item[c)] Utilizando un paso $h=0,1$ realizar una nueva tabla de valores a partir de la funci�n $f(x)$ dada. Obtener una aproximaci�n de $f'(x)$ en cada punto. Calcule el error cometido.

\end{itemize}



\end{document}
